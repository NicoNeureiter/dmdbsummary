% ------------------------------------------------------------------------------------------------ %
% INTRODUCTION
% ------------------------------------------------------------------------------------------------ %


\section{Introduction}
\renewcommand{\leftmark}{\oldleftmark}


% ------------------------------------------------------------------------------------------------ %
% MOTIVATION
% ------------------------------------------------------------------------------------------------ %


\subsection{Motivation}

A \emph{database management system} (DMBS) is used to
\begin{itemize}
\item avoid redundancy and inconsistency,
\item access the data in a declarative manner,
\item synchronize concurrent data accesses,
\item recover after a system failure,
\item guarantee security and privacy and
\item facilitate reuse of the data.
\end{itemize}


% ------------------------------------------------------------------------------------------------ %
% DATABASE ABSTRACTION LAYERS
% ------------------------------------------------------------------------------------------------ %


\subsection{Database Abstraction Layers}

\begin{figure}[htbp]
\begin{center}
\begin{tikzpicture}[>=latex']
	\node [draw, cylinder, shape border rotate=90, aspect=0.2, minimum width=3cm] (physical) at (0cm,-0.2cm) {physical layer};
	\node [draw, rectangle, minimum width=3cm, minimum height=0.7cm] (logical) at (0cm,1.5cm) {logical layer};
	\node [draw, rectangle, minimum height=0.7cm] (view1) at (-2cm,3cm) {view 1};
	\node [draw, rectangle, minimum height=0.7cm] (view2) at (-0.5cm,3cm) {view 2};
	\node [draw, rectangle, minimum height=0.7cm] (viewn) at (2cm,3cm) {view $n$};
	\node (dots) at (0.75cm,3cm) {$\cdots$};
	\node (phantom) at (0cm,0.15cm) {};
	\path [->] (phantom) edge (logical);
	\path (logical) edge (0cm,2.25cm);
	\path [->] (-2cm,2.25cm) edge (view1);
	\path [->] (-0.5cm,2.25cm) edge (view2);
	\path [->] (2cm,2.25cm) edge (viewn);
	\path (-2cm,2.25cm) edge (2cm,2.25cm);
\end{tikzpicture}
\end{center}
\caption[Database Abstraction Layers]{The three database abstraction layers.}
\end{figure}

There are three layers of abstraction:
\begin{itemize}
\item The \emph{physical layer} determines how to store data.
\item At the \emph{logical layer} the schema of the database determines which data is stored.
\item Whereas the schema is an integrated model of the whole information, \emph{views} offer only subsets of this information.
\end{itemize}


% ------------------------------------------------------------------------------------------------ %
% DATA INDEPENDENCE
% ------------------------------------------------------------------------------------------------ %


\subsection{Data Independence}

Due to the three layers of abstraction there are two kinds of data independence:

\begin{itemize}
\item \emph{Physical data independence} is fulfilled when modifications of the physical data structures do not affect the logical layer.
\item \emph{Logical data independence} is achieved by hiding smaller changes at the logical layer from the views.
\end{itemize}

Most of the present databases fulfill physical data independence. Logical data independence can only be guaranteed for minor modifications.


% ------------------------------------------------------------------------------------------------ %
% DATA MODELS
% ------------------------------------------------------------------------------------------------ %

\subsection{Data Modeling}

\begin{figure}[htbp]
\begin{center}
\begin{tikzpicture}[auto,>=latex']
	\node[draw, cloud, cloud puffs=15.7, aspect=2] (world) at (0cm,4cm) {mini world};
	\node[draw, rectangle] (schema) at (0cm,2cm) {\begin{tabular}{c} conceptual schema \\ (ER schema) \end{tabular}};
	\node[draw, rectangle, minimum width=2.5cm] (schema1) at (-3cm,0cm) {\begin{tabular}{c} relational \\ schema \end{tabular}};
	\node[draw, rectangle, minimum width=2.5cm] (schema2) at (0cm,0cm) {\begin{tabular}{c} object-oriented \\ schema \end{tabular}};
	\node[draw, rectangle, minimum width=2.5cm] (schema3) at (3cm,0cm) {\begin{tabular}{c} XML \\ schema \end{tabular}};
	\path [->] (world) edge (schema);
	\path [->] (schema) edge (schema1);
	\path [->] (schema) edge (schema2);
	\path [->] (schema) edge (schema3);
	\node () at (3.3cm,3cm) {manual modeling};
	\node () at (3.3cm,1.2cm) {\begin{tabular}{c} semi-automatic \\ transformation \end{tabular}};
\end{tikzpicture}
\end{center}
\caption[Data Modeling Phases]{Phases of data modeling.}
\end{figure}

There are three basic categories of models:
\begin{itemize}
\item The \emph{conceptual model} is a collection of entities and describes how they relate to each other. It captures the domain to be represented. Commonly used conceptual data models are the entity relationship model and UML.
\item The \emph{logical model} (schema) is a mapping of the concepts to a concrete logical representation. Some logical data models are the relational data model, the object-oriented data model and XML.
\item The \emph{physical model} is the implementation in a concrete hardware architecture.
\end{itemize}


% ------------------------------------------------------------------------------------------------ %