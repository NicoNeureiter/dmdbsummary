% ------------------------------------------------------------------------------------------------ %
% RELATIONAL MODEL
% ------------------------------------------------------------------------------------------------ %


\section{Relational Model}

\subsection{Definition of a Relational Model}


% ------------------------------------------------------------------------------------------------ %
% MATHEMATICAL FORMALISM
% ------------------------------------------------------------------------------------------------ %


\subsubsection{Mathematical Formalism}

Let $D_1, D_2, \ldots, D_n$ be \emph{domains}. These domains may only contain atomic values (e.g. strings and integers, but no sets). A \emph{relation} $R$ is a subset of the Cartesian product of $n$ domains: 
\[ \begin{array}{rcl}
R & \subseteq & D_1 \times D_2 \times \ldots \times D_n
\end{array} \]

An element of the set $t \in R$ is called a \emph{tuple}. Since a relation is defined as a set the tuples are unordered and there are no duplicates.

\begin{note}
The domains do not need to be different, i.e. $D_i = D_j$ is possible for $i \neq j$.
\end{note}


% ------------------------------------------------------------------------------------------------ %
% DEFINITION OF A SCHEMA
% ------------------------------------------------------------------------------------------------ %


\subsubsection{Definition of a Schema}

A \emph{schema} $\mc{R}$ of a relation $R$, sometimes denoted by $\mbf{sch}(R)$, is a set of \emph{attributes} $\{A_1, \ldots, A_n\}$ that assigns a name to every component of a tuple $t \in R$.

The names of the attributes have to be unique within a relation. As in the ER model a key is a minimal set of attributes that identify each tuple uniquely.

The domain of the attribute $A_i$ is denoted by $\mbf{dom}(A_i)$. Thus, a relation $R$ is a subset of the Cartesian product of the $n$ domains $\mbf{dom}(A_1), \mbf{dom}(A_2), \ldots, \mbf{dom}(A_n)$:
\[ \begin{array}{rcl}
R & \subseteq & \mbf{dom}(A_1) \times \mbf{dom}(A_2) \times \ldots \times \mbf{dom}(A_n)
\end{array} \]


\begin{example} An address book can be modeled as a relation
\[ \begin{array}{rcl}
\msf{addressbook} & \subseteq & \msf{string \times string \times integer}.
\end{array}\]
The schema assigns names to the components of a tuple and makes the relation a lot more readable:
\[
\msf{addressbook : \{ [ name:string, address:string, \underline{tel\#:integer}] \}}
\]
The square brackets $[\ldots]$ are a tuple constructor and the curly brackets $\{\ldots\}$ indicate that the relationship is a set of tuples.
\end{example}


% ------------------------------------------------------------------------------------------------ %
% FROM AN ER MODEL TO A RELATIONAL MODEL
% ------------------------------------------------------------------------------------------------ %


\subsection{From an ER Model to a Relational Model}


% ------------------------------------------------------------------------------------------------ %
% RELATIONAL REPRESENTATION OF ENTITY TYPES
% ------------------------------------------------------------------------------------------------ %


\subsubsection[Representation of Entity Types]{Relational Representation of Entity Types}

Every attribute of the entity becomes an attribute of the relation; keys are directly adopted and stay the same.

\begin{note}
The set of attributes of a weak entity's relational representation also includes the key of the entity it depends on.
\end{note}


% ------------------------------------------------------------------------------------------------ %
% RELATIONAL REPRESENTATION OF RELATIONSHIPS
% ------------------------------------------------------------------------------------------------ %


\subsubsection[Representation of Relationships]{Relational Representation of Relationships}

The relation $R$ representing the abstract $n$-ary relationship depicted in figure \ref{fig_gen_relationship} contains all attributes of the relationship and all key attributes of the entities involved.

\[
R : \{[
\underbrace{A_{1,1}, \ldots, A_{1,k_1}}_\text{key of $E_1$}, \ldots
\underbrace{A_{n,1}, \ldots, A_{n,k_n}}_\text{key of $E_n$},
\underbrace{A_1^R, \ldots, A_{k_R}^R}_\text{attributes of $R$}
]\}
\]

\begin{figure}[htbp]
\begin{center}
\begin{tikzpicture}

\erentity{e1}{-1.5cm,1.5cm}{$E_1$}
\erattributesmall{a11}{-3.5cm,2.7cm}{\underline{$A_{1,1}$}}
\node () at (-3.5cm,2cm){$\vdots$};
\erattributesmall{a12}{-3.5cm,1.1cm}{\underline{$A_{1,k_1}$}}
\erattributesmall{a13}{-3.5cm,0.3cm}{$\cdots$}

\node () at (0cm,1.5cm) {$\cdots$};

\erentity{en}{1.5cm,1.5cm}{$E_n$}
\erattributesmall{an1}{3.5cm,2.7cm}{\underline{$A_{n,1}$}}
\node () at (3.5cm,2cm){$\vdots$};
\erattributesmall{an2}{3.5cm,1.1cm}{\underline{$A_{n,k_1}$}}
\erattributesmall{an3}{3.5cm,0.3cm}{$\cdots$}

\errelation{r}{0cm,0cm}{$R$}
\erattributesmall{ar1}{-1.3cm,-1cm}{$A_1^R$}
\node () at (0cm,-1cm) {$\cdots$};
\erattributesmall{ark}{1.3cm,-1cm}{$A_{k_R}^R$}
\path (r) edge (e1);
\path (e1) edge (a11);
\path (e1) edge (a12);
\path (e1) edge (a13);
\path (r) edge (en);
\path (en) edge (an1);
\path (en) edge (an2);
\path (en) edge (an3);
\path (r) edge (ar1);
\path (r) edge (ark);
\end{tikzpicture}
\end{center}
\caption[n-ary Relationship]{Example of a generic $n$-ary relationship.}
\label{fig_gen_relationship}
\end{figure}

\begin{example}
The relationship depicted in figure \ref{fig_er_model} translates to:
\[
\msf{attend : \{[\underline{id:integer, number:integer}]\}}
\]
Since this is a N:M relationship \textit{id} as well as \textit{number} are part of the key.
\end{example}

In some situations relations with the same can be summarized. This works relations of 1:1 and 1:N relationships.


% ------------------------------------------------------------------------------------------------ %
% RELATIONAL ALGEBRA
% ------------------------------------------------------------------------------------------------ %


\subsection{Relational Algebra}


% ------------------------------------------------------------------------------------------------ %
% SELECTION
% ------------------------------------------------------------------------------------------------ %


\subsubsection[Selection]{Selection ($\sigma$)}

A \emph{selection} is a unary operation $\sigma_{A \circ  B}(R)$ or $\sigma_{A \circ v}(R)$, where $R$ is a relation, $A$ and $B$ are attribute names, $v$ is a constant value and $\circ \in \{=,<,\leq,>,\geq,\neq\}$ is a comparison.

The result of the selection $\sigma_{A \circ B}(R)$ contains all tuples in $R$ for which $\circ$ holds between the attributes $A$ and $B$:
\[ \begin{array}{rcl}
\sigma_{A \circ B}(R) & = & \{ t \in R \mid t.A \circ t.B \}
\end{array} \]

The result of the selection $\sigma_{A \Phi v}(R)$ contains all tuples in $R$ for which $\Phi$ holds between the attribute $A$ and the constant value $v$:
\[ \begin{array}{rcl}
\sigma_{A \Phi V}(R) & = & \{ t \in R \mid t.A \Phi v \}
\end{array} \]

A \emph{generalized selection} is a unary operation $\sigma_\phi(R)$, where $R$ is a relation and $\phi$ is a propositional formula that consists of atoms as allowed in the normal selection and the logical operators $\wedge$, $\vee$, and $\neg$. The result of this selection contains all tuples in $R$ for which $\phi$ holds:
\[ \begin{array}{rcl}
\sigma_{\phi_1 \wedge \phi_2} & = & \sigma_{\phi_1}(R) \cap \sigma_{\phi_2}(R) \\[0.5em]
\sigma_{\phi_1 \vee \phi_2} & = & \sigma_{\phi_1}(R) \cup \sigma_{\phi_2}(R) \\[0.5em]
\sigma_{\neg \phi} & = & R \setminus \sigma_{\phi}(R) 
\end{array} \]


% ------------------------------------------------------------------------------------------------ %
% PROJECTION
% ------------------------------------------------------------------------------------------------ %


\subsubsection[Projection]{Projection ($\pi$)}

The \emph{projection} operator $\pi_{A_1, \ldots, A_n}$ is defined as
\[ \begin{array}{rcl}
\pi_{A_1, \ldots, A_n}(R) & = & \{ t[A_1, \ldots, A_n] \mid t \in R \},
\end{array} \]
where $t[A_1,\ldots,A_n]$ is the tuple $t$ restricted to the attributes $A_1, \ldots, A_n$.


% ------------------------------------------------------------------------------------------------ %
% CARTESIAN PRODUCT
% ------------------------------------------------------------------------------------------------ %


\subsubsection[Cartesian Product]{Cartesian Product ($\times$)}

The  \emph{Cartesian product} $R \times S$ contains all $|R| \cdot |S|$ possible combinations of tuples in the relations $R$ and $S$:
\[ \hspace{-0.5em} \begin{array}{ll}
R \times S & = \\
\multicolumn{2}{l}{\{ [r_1,\ldots,r_n,s_1,\ldots,s_m] \mid [r_1,\ldots,r_n] \in R, [s_1,\ldots,s_m] \in S \}}
\end{array} \hspace{-0.5em} \]

\begin{note}
The Cartesian product is defined differently from the one in set theory. The Cartesian product of an $n$-tuple by an $m$-tuple is flattened into an $(n+m)$-tuple.
\end{note}

The schema of the Cartesian product $R \times S$ is
\[ \begin{array}{rclcl}
\mbf{sch}(R \times S) & = &
\mbf{sch}(R) \cup \mbf{sch}(S) & = &
\mc{R} \cup \mc{S}.
\end{array} \]
If the attribute $A$ appears in $\mc{R}$ as well as in $\mc{S}$, they are renamed as $R.A$ and $R.B$; otherwise their names would not be unique in $\mc{R} \cup \mc{S}$.


% ------------------------------------------------------------------------------------------------ %
% RENAME
% ------------------------------------------------------------------------------------------------ %


\subsubsection[Rename Operator]{Rename Operator ($\rho$)}

The operator $\rho_{R'}(R)$ \emph{renames} the relation $R$ as $R'$. And the operator $\rho_{A' \la A}$ that renames a single attribute is defined as
\[ \begin{array}{rcl}
\rho_{A' \la A}(R) & = & \{t[A' \la A] \mid t \in R\},
\end{array} \]
where $t[A' \la A]$ is the tuple $t$ with the attribute $A$ renamed as $A'$.

\begin{example} In some cases renaming is necessary. The following combination of operators returns the name of all persons and the name the persons' fathers:
\[\pi_{P_1.\msf{name},P_2.\msf{name}}(\sigma_{P_1.\msf{father}=P_2.\msf{id}}(\rho_\msf{P_1}(\msf{person}) \times \rho_\msf{P_2}(\msf{person})))\]
\end{example}


% ------------------------------------------------------------------------------------------------ %
% SET OPERATIONS
% ------------------------------------------------------------------------------------------------ %


\subsubsection[Set Operators]{Set Operators ($\cup$, $\cap$ and $\setminus$)}

Let $R$ and $S$ be two relations with the same schema (i.e. $\mc{R} = \mc{S})$. Just like in set theory, the \emph{union}, \emph{intersection} and \emph{difference} operators are
\[ \begin{array}{rcl}
R \cup S & = & \{ t \mid t \in R \vee t \in S \} \\[0.5em]
R \cap S & = & \{ t \mid t \in R \wedge t \in S \} \\[0.5em]
R \setminus S & = & \{ t \mid t \in R \wedge t \notin S \}.
\end{array} \]

% \begin{note} The intersection $R \cup S$ can be expressed by two differences:
% \[ \begin{array}{rcl}
% R \cup S & = & R \setminus (R \setminus S)
% \end{array} \]
% \end{note}


% ------------------------------------------------------------------------------------------------ %
% JOIN OPERATORS
% ------------------------------------------------------------------------------------------------ %


\subsubsection{Join Operators}
\label{join}

There is a variety of different \emph{join} operators:

\newcommand{\lefttable}{\begin{array}{|c|c|}\hline\multicolumn{2}{|c|}{R} \\\hline A & B \\\hline\hline	a_1 & b_1 \\ a_2 & b_2 \\\hline \end{array}}
\newcommand{\righttable}{\begin{array}{|c|c|} \hline\multicolumn{2}{|c|}{S} \\\hline B & C \\\hline\hline b_1 & c_1 \\ b_3 & c_2 \\\hline \end{array}}

\begin{itemize}
\item The \emph{theta join} $R \join_\theta S$, where $\theta$ is an arbitrary join predicate, can be defined as
	\[ \begin{array}{rcl}
	R \join_\theta S & = & \sigma_\theta(R \times S).
	\end{array} \]
\item The \emph{natural join} of two relations $R$ and $S$ is denoted by $R \join S$. Without loss of generality $R$ is assumed to have $n+k$ attributes $A_1,\ldots,A_n,B_1,\ldots,B_k$ and $S$ the $m+k$ attributes $B_1,\ldots,B_k,C_1,\ldots,C_m$ (where all $A_i$ and $C_j$ are pairwise different). The natural join can be defined as
	\[ \begin{array}{rcl}
	R \join S & =  & \pi_{\mc{R} \cup \mc{S}}(\sigma_\theta(R \times S)),
	\end{array} \]
	where $\theta \equiv \bigwedge_{i=1}^k R.B_i = S.B_i$.
	\[
	\begin{array}{ccccc}
	\lefttable & \join & \righttable
	& = &
	\begin{array}{|c|c|c|} \hline
	\multicolumn{3}{|c|}{R \join S} \\\hline
	A & B & C \\\hline\hline
	a_1 & b_1 & c_1 \\\hline
	\end{array}
	\end{array}
	\]
\item The \emph{left outer join} $R \leftouterjoin S$ is like the natural join, but all tuples of the left argument's relation $R$ are preserved.
	\[
	\begin{array}{ccccc}
	\lefttable & \leftouterjoin & \righttable
	& = &
	\begin{array}{|c|c|c|} \hline
	\multicolumn{3}{|c|}{R \leftouterjoin S} \\\hline
	A & B & C \\\hline\hline
	a_1 & b_1 & c_1 \\
	a_2 & b_2 & - \\\hline
	\end{array}
	\end{array}
	\]
\item The \emph{right outer join} $R \rightouterjoin S$ is analogous to the left outer join.
	\[
	\begin{array}{ccccc}
	\lefttable & \rightouterjoin & \righttable
	& = &
	\begin{array}{|c|c|c|} \hline
	\multicolumn{3}{|c|}{R \rightouterjoin S} \\\hline
	A & B & C \\\hline\hline
	a_1 & b_1 & c_1 \\
	- & b_3 & c_2 \\\hline
	\end{array}
	\end{array}
	\]
\item The \emph{outer join} $\leftouterjoin$ can be defined as
	\[ \begin{array}{rcl}
	R \outerjoin S & = & (R \leftouterjoin S) \cup (R \rightouterjoin S).
	\end{array} \]
	In this sense the outer join combines the left and right outer join.
	\[
	\begin{array}{ccccc}
	\lefttable & \outerjoin & \righttable
	& = &
	\begin{array}{|c|c|c|} \hline
	\multicolumn{3}{|c|}{R \outerjoin S} \\\hline
	A & B & C \\\hline\hline
	a_1 & b_1 & c_1 \\
	a_2 & b_2 & - \\
	- & b_3 & c_2 \\\hline
	\end{array}
	\end{array}
	\]
\item The \emph{left semi join}
	\[ \begin{array}{rcl}
	R \leftsemijoin S & = & \pi_\mc{R}(R \join S)
	\end{array} \]
	is the natural join of $R$ and $S$ projected onto the attributes of $R$.
	\[
	\begin{array}{ccccc}
	\lefttable & \leftsemijoin & \righttable
	& = &
	\begin{array}{|c|c|c|} \hline
	\multicolumn{2}{|c|}{R \leftsemijoin S} \\\hline
	A & B \\\hline\hline
	a_1 & b_1 \\\hline
	\end{array}
	\end{array}
	\]
\item The \emph{right semi join} $R \rightsemijoin S$ is analogous to the left semi join.
	\[
	\begin{array}{ccccc}
	\lefttable & \rightsemijoin & \righttable
	& = &
	\begin{array}{|c|c|c|} \hline
	\multicolumn{2}{|c|}{R \rightsemijoin S} \\\hline
	B & C \\\hline\hline
	b_1 & c_1 \\\hline
	\end{array}
	\end{array}
	\]
\end{itemize}


% ------------------------------------------------------------------------------------------------ %
% RELATIONAL DIVISION
% ------------------------------------------------------------------------------------------------ %


\subsubsection[Relational Division]{Relational Division ($\div$)}

Let $R$ and $S$ be two relations such tat $\mc{S} \subseteq \mc{R}$. The result of the \emph{relational division} $R \div S$ consists of the restrictions of tuples in $R$ to $\mc{R} \setminus \mc{S}$, for which it holds that all their combinations with tuples in $S$ are represented in $R$.

\[ \begin{array}{ccccc}
\begin{array}{|c|c|}\hline
\multicolumn{2}{|c|}{R} \\\hline
A & B \\\hline\hline
a_1 & b_1 \\
a_1 & b_2 \\
a_1 & b_3 \\
a_2 & b_2 \\
a_2 & b_3 \\\hline
\end{array}
& \div &
\begin{array}{|c|}\hline
S \\\hline
B \\\hline\hline
b_1 \\
b_2 \\\hline
\end{array}
& = &
\begin{array}{|c|}\hline
R \div S \\\hline
A \\\hline\hline
a_1 \\\hline
\end{array}
\end{array} \]

The relational division can be simulated with basic operators as follows:
\[ \begin{array}{rcl}
R \div  S & = & \pi_{\mc{R}\setminus\mc{S}}(R) \setminus \pi_{\mc{R}\setminus\mc{S}}((\pi_{\mc{R}\setminus\mc{S}}(R) \times S) \setminus R)
\end{array} \]


% ------------------------------------------------------------------------------------------------ %
% RELATIONAL CALCULUS
% ------------------------------------------------------------------------------------------------ %


\subsection{Relational Calculus}

The \emph{relational calculus} is based on the first-order predicate calculus. There are two equally powerful relational calculi:

\begin{itemize}
\item tuple relational calculus and
\item domain relational calculus.
\end{itemize}


% ------------------------------------------------------------------------------------------------ %
% TUPLE RELATIONAL CALCULUS
% ------------------------------------------------------------------------------------------------ %


\subsubsection{Tuple Relational Calculus}

Expressions in \emph{tuple relational calculus} are of the form
\[
\{ t \mid P(t) \}
\quad\text{or}\quad
\{ [t_1.A_1,\ldots,t_n.A_n \mid P(t_1,\ldots,t_n) \}
,\]
where $t,t_1,\ldots,t_n$ are tuple variable (note that $t_i=t_j$ is possible for $i\neq j$), $A_1,\ldots,A_n$ are attribute names and $P(t)$ is a predicate formula. Atoms of such a formula are
\begin{itemize}
\item $u \in R$, where $u$ is a tuple variable and $R$ is a name of a relation and
\item $u.A \circ v.B$ or $u.A \circ c$, where $u$ and $v$ are tuple variables, $A$ and $B$ attribute names, $\circ \in \{=,<,\leq,>,\geq,\neq\}$ a comparison and $c$ is a constant.
\end{itemize}

Formulae are constructed by the following rules:
\begin{itemize}
\item
	All atoms are formulae.
\item
	If $P$ is a formula, then $\neg P$ and $(P)$ are formulae too.
\item
	If $P_1$ and $P_2$ are formulae, then $P_1 \wedge P_2$, $P_1 \vee P_2$ and $P_1 \RA P_2$ are formulae too.
\item
	If $P(t)$ is a formula with a free variable $t$, then
	\[
	\forall t \in R(P(t))
	\qquad\text{and}\qquad
	\exists t \in R(P(t))
	\]
are formulae too; note that the tuple variable $t$ of the quantifications is bound to the relation $R$.
\end{itemize}

\begin{example} The expression
\[ \{
s \mid
s \in \msf{student} \wedge
\neg (\exists a \in \msf{attend}(s.\msf{id}=a.\msf{id}))
\} \]
lists all students that are attending no lectures.
\end{example}


% ------------------------------------------------------------------------------------------------ %


\subsubsection[Safe Expressions in TRC]{Safe Expressions in Tuple Relational Calculus}

Some expressions in tuple relational calculus yield infinite results. An example of such an expression is
\[ \{ t \mid \neg(t \in R) \}. \]

This is not desirable. An expression is considered \emph{safe} if its result is a subset of the domain of the formula. The domain of a formula is the set of all constants and the relations' values that appear in the formula. The result of a safe expression is always finite.


% ------------------------------------------------------------------------------------------------ %
% DOMAIN RELATIONAL CALCULUS
% ------------------------------------------------------------------------------------------------ %


\subsubsection{Domain Relational Calculus}

Expressions of \emph{domain relational calculus} are of the form
\[ \{[v_1,\ldots,v_n] \mid P(v_1,\ldots,v_n) \}, \]
where each $v_i$ (for $1 \leq i \leq n$) is a domain variable and $P(v_1,\ldots,v_n)$ denotes a predicate over the free variables $v_1,\ldots,v_n$.

Atoms of formulae in domain relational calculus are
\begin{itemize}
\item $[w_1,\ldots,w_m] \in R$, where $R$ is an $m$-ary relation and
\item $x \circ y$ and $x \circ c$, where $x$ and $y$ are domain variables, $c$ is a constant and $\circ \in \{=,<,\leq,>,\geq,\neq\}$ is a comparison.
\end{itemize}

Formulae are constructed according to the following rules:
\begin{itemize}
\item
	An atom is a formula.
\item
	If $P$ is a formula, then $\neg P$ and $(P)$ are formulae too.
\item
	If $P_1$ and $P_2$ are formulae, then $P_1 \wedge P_2$, $P_1 \vee P_2$ and $P_1 \RA P_2$ are formulae too.
\item
	If $P(v)$ is a formula with a free variable $t$, then
	\[ \forall v (P(v))
	\qquad\text{and}\qquad
	\exists v (P(v)) \]
	are formulae too.
\end{itemize}

\begin{example}
The expression
\[ \begin{array}{rcl} \{
[i,n] & \mid & \exists s ([i,n,s] \in \msf{student} \\
& \wedge & \exists n' ([i,n'] \in \msf{attends} \\
& \wedge & \exists t,c ([n',t,c] \in \msf{lecture} \\
& \wedge & c \geq 8 ))) \}
\end{array} \]
lists the id and name of all students that attend a lecture with more at least $8$ credits (compare with ER model in figure \ref{fig_er_model}).
\end{example}


% ------------------------------------------------------------------------------------------------ %
% SAFE EXPRESSIONS IN DOMAIN RELATIONAL CALCULUS
% ------------------------------------------------------------------------------------------------ %


\subsubsection[Safe Expressions in DRC]{Safe Expressions in Domain Relational Calculus}

Like in the tuple relational calculus the domain relational calculus allows infinite results:
\[
\{ [v_1,\ldots,v_n] \mid \neg([v_1,\ldots,v_n] \in R) \}
\]

An expression $\{[v_1,\ldots,v_n\mid P(v_1,\ldots,v_n)\}$ is \emph{safe}, if the following conditions hold:
\begin{itemize}
\item If the tuple $[v_1,\ldots,v_{i-1},c_i,v_{i+1},\ldots,v_n]$, where $c_i$ is a constant, is contained in the result, then $c_i$ must be contained in the domain of $P$.
\item Every existentially quantified subformula $\exists v(P'(v))$ may only accept elements of the domain of $P'$. In other words: If $P'(c)$ holds for a constant value, then $c$ is contained in the domain of $P'$.
\item Every universally quantified subformula $\forall v(P'(v))$ holds, if and only if $P'(v)$ holds for every value in the domain of $P'$. In other words: $P'(c)$ holds for all values $c$ that are not contained in the domain of $P'$.
\end{itemize}


% ------------------------------------------------------------------------------------------------ %
% CODD'S THEOREM
% ------------------------------------------------------------------------------------------------ %


\subsection{Codd's Theorem}

\emph{Codd's theorem} states that
\begin{itemize}
\item relational algebra,
\item tuple relational calculus restricted to safe expressions and
\item domain relational calculus restricted to safe expressions
\end{itemize}
are precisely equivalent in expressive power. That is, a database query can be formulated in one language if and only if it can be expressed in the others.

Query languages that are equivalent in expressive power to relational algebra are called \emph{relationally complete}.


% ------------------------------------------------------------------------------------------------ %