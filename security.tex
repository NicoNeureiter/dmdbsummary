% ------------------------------------------------------------------------------------------------ %
% SECURITY
% ------------------------------------------------------------------------------------------------ %


\section{Database Security}


% ------------------------------------------------------------------------------------------------ %
% SECURITY TASKS
% ------------------------------------------------------------------------------------------------ %


\subsection{Security Tasks}

There are three kinds of security mechanisms in a DBMS:
\begin{itemize}
\item \emph{Authentication:} Verifying the identification of a user.
\item \emph{Authorization:} Checking the access privileges.
\item \emph{Auditing:} Looking for violations (in the past).
\end{itemize}
Data security consists of confidentality and integrity.


% ------------------------------------------------------------------------------------------------ %
% DISCRETIONARY ACCESS CONTROL
% ------------------------------------------------------------------------------------------------ %


\subsection{Discretionary Access Controls}

Access rules of \emph{discretionary acces controls (DAC)} assign access rights $t$ for an object $o$ to a subject $s$. Formally, access rules are quintuples $(o,s,t,p,f)$, where
\begin{itemize}
\item $o \in O$ and $O$ is the set of \emph{objects} (e.g. tables, tuples, attributes),
\item $s \in S$ and $S$ is the set of \emph{subjects} (e.g. users, processes, applications),
\item $t \in T$ and $T$ is the set of \emph{acces rights} (e.g. read, write, delete),
\item $p$ is a predicate (e.g. level = `c4') and
\item $f$ is a boolean value specifying, whether $s$ may grant the privilege $(o,t,p)$ to another subject $s' \in S$.
\end{itemize}

A simple way to store the access rules is a so called acces matrix. Other ways to control accesses are views or query rewriting.


% ------------------------------------------------------------------------------------------------ %
% ACCESS CONTRONL IN SQL
% ------------------------------------------------------------------------------------------------ %


\subsubsection{Access Control in SQL}


\begin{lstlisting}[language=sql,morekeywords={to}]
grant update (id, name, semester)
   on student
   to peter;
\end{lstlisting}

\begin{lstlisting}[language=sql,morekeywords={to}]
create view first_semester as
   select *
   from student
   where semester = 1;
grant select
   on first_semester
   to tutor;
\end{lstlisting}

Personal records can be protected by aggregation.

\clearpage



% ------------------------------------------------------------------------------------------------ %