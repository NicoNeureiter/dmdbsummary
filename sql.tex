% ------------------------------------------------------------------------------------------------ %
% SQL
% ------------------------------------------------------------------------------------------------ %


\section{SQL}

The language \emph{SQL (Structured Query Language)} is designed for managing data in relational DMBS. It is based upon relational algebra and tuple relational calculus. SQL is a family of standards:

\begin{itemize}
\item Data definition language (DDL)
\item Data mnipulation language (DML)
\item Query Language
\end{itemize}

\begin{note}
In SQL relations are referred to as tables and may contain duplicates.
\end{note}


% ------------------------------------------------------------------------------------------------ %
% DATA TYPES
% ------------------------------------------------------------------------------------------------ %


\subsection{Data Types}

Some of SQL's data types are:
\begin{itemize}
\item \emph{char($n$):} strings of length $n$, padded with spaces.
\item \emph{varchar($n$):} strings of variable length with a maximum size of $n$ characters.
\item \emph{integer:} for interger values.
\item \emph{numeric($p$,$s$):} numbers with $p-s$ digits before the decimal and $s$ digits after the decimal.
\item \emph{blob, raw}: large binaries.
\item \emph{clob}: large string values.
\item \emph{date}: dates.
\end{itemize}


% ------------------------------------------------------------------------------------------------ %
% DATA DEFINITION LANGUAGE
% ------------------------------------------------------------------------------------------------ %


\subsection{Data Definition Language}

A \emph{data definition language (DDL)} is a syntax for defining or modifying data structures, especially database schemas. The following SQL statement illustrates, how a new table is created:
\begin{lstlisting}[language=sql]
create table student
  (id   integer     primary key,
   name varchar(20) not null),
   age  integer;
\end{lstlisting}

Columns of an existing table can be added and altered:
\begin{lstlisting}[language=sql]
alter table student
   add column semester integer;
alter table student
   alter column name varchar(30);
\end{lstlisting}

Similarly, a column is removed with \lstinline[language=sql]{drop column}. Indexes is created and deleted like this:

\begin{lstlisting}[language=sql]
create index my_index on
   student(name, age);
drop index my_index;
\end{lstlisting}


% ------------------------------------------------------------------------------------------------ %
% DATA MANIPULATION LANGUAGE
% ------------------------------------------------------------------------------------------------ %


\subsection{Data Manipulation Language}

A \emph{data manipulation language (DML)} is used to insert, update and delete tuples:
\begin{lstlisting}[language=sql]
insert into student (id, name)
   values (123, 'L. Skywalker');
update student
   set semester = semester+1;
delete student
   where age < 13;
\end{lstlisting}


% ------------------------------------------------------------------------------------------------ %
% QUERY
% ------------------------------------------------------------------------------------------------ %


\subsection{Queries}


\begin{lstlisting}[language=sql]
select [distinct] columns
   from tablename [as alias]
   [where condition]
   [group by (attribute)+]
   [having condition]
   [order by (attribute [asc|desc])+]
\end{lstlisting}

\begin{itemize}
\item The \emph{distinct} keywords eliminates all duplicate tuples.
\item Results are sorted using the \emph{order by} key words.
\end{itemize}

% ------------------------------------------------------------------------------------------------ %


\begin{center}
\begin{tabular}{|p{4em}|p{4em}|}\hline
$v$ & \bf not $v$ \\\hline
true       & false   \\
unknown    & unknown \\
false      & true    \\\hline
\end{tabular}
\end{center}

\begin{center}
\begin{tabular}{|p{4em}|p{4em}p{4em}p{4em}|}\hline
\bf and & true    & unknown & false   \\\hline
true    & true    & unknown & false   \\
unknown & unknown & unknown & false   \\
false   & false   & false   & false   \\\hline
\end{tabular}
\end{center}

\begin{center}
\begin{tabular}{|p{4em}|p{4em}p{4em}p{4em}|}\hline
\bf or  & true    & unknown & false   \\\hline
true    & true    & true    & true    \\
unknown & true    & unknown & unknown \\
false   & true    & unknown & false   \\\hline
\end{tabular}
\end{center}

% ------------------------------------------------------------------------------------------------ %


\subsection{Isolation in SQL92}

Isolation in a database determines how and when changes of a transaction becom visible for other transactions.

\begin{lstlisting}[morekeywords={set,transaction,isolation,level,diagnostic,size}]
set transaction
   [read only, | read write,]
   [isolation level
      read uncommited, |
      read commited,   |
      repeatable read, |
      serializable,]
   [diagnostic size ...,]
\end{lstlisting}

\begin{itemize}
\item \emph{Read uncommited:} This is the lowest level of isolation and is only allowed for read only transactions. Reads do not need a shared lock and are not hindered by exclusive locks.
\item \emph{Read commited:} Transactions only read commited versions of objects. However, one transaction may read different versions of an object.
\item \emph{Repeatable read} Reading different versions of the same object is prevented, but phantoms can still happen.
\item \emph{Serializable:} Full isolation is guaranteed.
\end{itemize}


% ------------------------------------------------------------------------------------------------ %